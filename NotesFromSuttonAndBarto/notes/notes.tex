\documentclass[11pt,a4paper]{article}
\pdfoutput=1

\usepackage{geometry}                % See geometry.pdf to learn the layout options. There are lots.
\geometry{letterpaper}                   % ... or a4paper or a5paper or ... 
%\geometry{landscape}                % Activate for for rotated page geometry
%\usepackage[parfill]{parskip}    % Activate to begin paragraphs with an empty line rather than an indent
\usepackage{graphicx}
\usepackage{amssymb}
%\usepackage{epstopdf}
%\DeclareGraphicsRule{.tif}{png}{.png}{`convert #1 `dirname #1`/`basename #1 .tif`.png}
\usepackage{xcolor}
\usepackage{amsmath}
\usepackage{subfigure}
\usepackage{listings}    
\usepackage{bbm}
\usepackage{cancel}
\usepackage{relsize}
%\usepackage{listings}
\usepackage{xcolor}
%\usepackage{tcolorbox}
\usepackage[most]{tcolorbox}

\definecolor{codegreen}{rgb}{0,0.6,0}
\definecolor{codegray}{rgb}{0.5,0.5,0.5}
\definecolor{codepurple}{rgb}{0.58,0,0.82}
\definecolor{backcolour}{rgb}{0.95,0.95,0.92}


%%%%%%% Borun's commands %%%%%%%%%%%%%%

\newcommand\be{\begin{equation}}
\newcommand\bea{\begin{eqnarray}}
\newcommand\ee{\end{equation}}
\newcommand\eea{\end{eqnarray}}
\newcommand\Regge{\alpha'}
\newcommand{\bdm}{\begin{displaymath}}
\newcommand{\edm}{\end{displaymath}}
\newcommand{\nn}{\nonumber \\}
\newcommand{\f}[2]{\frac{#1}{#2}}
\newcommand{\bref}[1]{(\ref{#1})}
\newcommand\h{\frac{1}{2}}
\newcommand{\ket}[1]{|#1 \rangle}
\newcommand{\bra}[1]{\langle #1 |}
\newcommand{\e}[1]{\mathbb E[#1]}
\newcommand{\todo}[1]{{\color{cyan}{TODO: #1}}}




\title{Reinforcement Learning}
\author{Borun D. Chowdhury}
%\date{}                                           % Activate to display a given date or no date

\begin{document}
\maketitle

\section{Bandits}


\subsection{Value Based Models}

Suppose there is a process where one can choose among a set of actions $\{a\}$ that give rewards $R$ then it makes sense to take actions that maximize the reward
\bea
q_*(a) = \e{R|A=a}
\eea
which is the expectation of the reward given the action chose is $a$.

We want to 
\begin{itemize}
\item Find the true values for each action
\be
Q_t(a) := \frac{\sum_{i=1}^{t-1} R_i \delta_{A_i,a}}{\sum_{i=1}^{t-1}  \delta_{A_i,1}} \label{rewardrunningmean}
\ee
\item Take the action with the max value
\be
A_t := \underset{a}{\text{argmax}} ~ Q_t(a)
\ee
\end{itemize}

However, the tricky part is to do them together. For this we can 
\begin{itemize}
\item Take an $\epsilon$-greedy approach where we exploit with probability $1-\epsilon$ and explore with probability $\epsilon$,
\item Use the upper-confidence-bound method (\todo{find the math behind this})
\be
A_t := \underset{a}{\text{argmax}} ~ \left[ Q_t(a) + c \sqrt{\frac{\ln(t)}{N_t(a)}}\right]
\ee
\end{itemize}

The computation of the running mean for rewards eqn \ref{rewardrunningmean} per action naively requires keeping track of all rewards but we can instead do
\be
Q_{n+1} = Q_n + \frac{1}{n} [R_n - Q_n]
\ee
Furthermore if the problem is non-stationary we can instead have a fixed parameter $\alpha$ to exponential weight the prior reward
\bea
Q_{n+1} &=& Q_n + \alpha [R_n - Q_n] \nn
&=& (1-\alpha) Q_n + \alpha R_n \nn
&=& \alpha R_n + (1-\alpha) \alpha R_{n-1} + (1-\alpha)^2 \alpha R_{n-2}  + \dots \nn
&=& (1-\alpha)^n Q_1 + \alpha \sum_{i=1}^n (1-\alpha)^{n-i} R_i
\eea

\section{Markov Reward Process}

A Markov Reward Process is a process where one starts off in a state and at each time makes a transition to another state (possibly the same) while earning some rewards
\be
(S_0,\cancel{R_0}),(S_1,R_1), \dots
\ee

The probability to get to state $S_T$ at time step $T$ is
\be
p(S_T) = \sum_{S_0} p(S_T|S_0)p(S_0)
\ee
Usually the initial state is sharply peaked $p(S_0) = \delta_{S_0,s_0}$ for some $s_0$. We have
\bea
\sum_{S_T} p(S_T|S_0) &=& 1 \\
p(S_T|S_0) &=& \sum_{S_t} p(S_T|S_t)p(S_t,S_0)
\eea

We then have
\bea
E[R_{t+1}|S_0] &=& \sum_{S_T,S_{t},R_{t+1}} R_{t+1} p(S_T,S_{t}) {\color{cyan} p(R_{t+1},S_{t})} p(S_{t},S_0) \nn
&=& \sum_{S_{t},R_{t+1}} R_{t+1} {\color{cyan} p(R_{t+1},S_{t})} p(S_{t},S_0)
\eea
The states after time $t$ do not matter for the expectation value of $R_T$ due to causality.



\section{Markov Decision Process}

Where Markov Decision Process also involves and agent choosing actions at each step
\be
(S_0,A_0,\cancel{R_0}),(S_1,A_1,R_1), \dots
\ee

Note that
\bea
p(S_T|S_0) &=& \sum_{S_t,S_0} p(S_T|S_t) p(S_t|S_0) \\
\sum_{S_T} p(S_T|S_t) &=& 1
\eea
Furthermore,
\bea
p(S_T|S_0) &=& \sum_{S_t,A_t} p(S_T|S_t,A_t) \pi(A_t|S_t) p (S_t|S_0) \\
\sum_{S_T} p(S_T|S_t,A_t) &=& 1
\eea

Thus we have
\bea
\mathbb E[R_{t+1}] &=&  \sum_{S_0,S_T,S_t,A_t} R_{t+1} p(S_T|S_t,A_t) {\color{cyan} p(R_{t+1}|S_t,A_t)} \pi(A_t|S_t) p (S_t|S_0) {\color{blue} p(S_0)} \nn
&=& \sum_{S_0,S_t,A_t} R_{t+1} {\color{cyan} p(R_{t+1}|S_t,A_t)} \pi(A_t|S_t) p (S_t|S_0) {\color{blue} p(S_0)}
\eea
Thus the states and actions after $t$ do not matter.

We can break the "propagator" into individual time steps so
\bea
\mathbb E[ R_{t+1}] &=& \sum_{R_{t+1}, S_0, S_1, \dots S_T, A_0,A_1, A_{T-1}} R_{t+1} \Bigg[ \nn
&& \qquad \color{red}{\pi(A_0|S_0)} \color{blue}{p(S_0)} \nn
&& \qquad \times \pi(A_{1}|S_{1}) p(S_{1}|S_0,A_0) \nn
&& \qquad \qquad \dots \nn
&&  \qquad \times \pi(A_{t}|S_{t}) p(S_{t}|S_{t-1},A_{t-1}) \nn
&& \qquad \times {\color{olive} \pi(A_{t+1}|S_{t+1}) p(S_{t+1}|S_{t},A_{t}) } \nn
&& \qquad \qquad {\color{olive} \dots} \nn
&& \qquad \times {\color{olive} \pi(A_{T-1}|S_{T-1}) p(S_{T-1}|S_{T-2},A_{T-2}) } \nn
&& \qquad \times {\color{olive} p(S_T)} \nn
&& \Bigg]  \nn
&& \times{\color{cyan} p(R_{t+1}| S_{t},A_{t})} \nn
&=& \sum_{R_{t+1}, S_0, S_1, \dots S_t, A_0,A_1, A_{t}} R_{t+1} \Bigg[ \nn
&& \qquad \color{red}{\pi(A_0|S_0)} \color{blue}{p(S_0)} \nn
&& \qquad \times \pi(A_{1}|S_{1}) p(S_{1}|S_0,A_0) \nn
&& \qquad \qquad \dots \nn
&&  \qquad \times \pi(A_{t}|S_{t}) p(S_{t}|S_{t-1},A_{t-1}) \nn
&& \Bigg]  \nn
&& \times{\color{cyan} p(R_{t+1}| S_{t},A_{t})}
\eea
Here the olive colored terms are redundant in that the sum on them can be done directly and will give one. 

We define a discounted (stochastic) reward as
\bea
G_t &:=& \sum_{k=0}^\infty \gamma^k R_{t+1+k} \nn
&=& R_{t+1} + \gamma G_{t+1}
\eea

The state value is defined as
\bea
v_\pi(s) &=& \mathbb E_\pi[G_0|S_0=s] \nn
&=& \sum_{t=0}^\infty \gamma^t \mathbb E_\pi[R_{t+1} |S_0=s]
\eea
Thus, we see why we had the term $\color{blue}{p(S_0)}$ in blue as to compute state values we take this probability to be sharply peaked while summing over all subsequent paths.

Similarly, the action-value is defined as 
\bea
q_\pi(s,a) &=& \mathbb E_\pi[G_0|S_0=s,A_0=a] \nn
&=& \sum_{t=0}^\infty \gamma^t \mathbb E_\pi[R_{t+1} |S_0=s,A_0=a]
\eea
and we see why we had the term $\color{red}{\pi(A_0|S_0)}$ in red as in this case this term is also sharply peaked and we only sum over subsequent steps. 

Since the process is Markovian the transition probabilities are independent of the time step. We define
\bea
p(s'|s,a) = p(S_{t+1}=s'|S_t=s,A_t=a)
\eea
and likewise other expressions.

We then get
\bea
v_\pi(s) &=& \e{G_t|S_t=s;\pi} \nn
&=& \e{R_{t+1} + \gamma G_{t+1} |S_t=s;\pi} \nn
&=& \e{R_{t+1}|S_t=s;\pi} + \gamma \e{G_{t+1}|S_t=s;\pi} \nn
&=& \e{R_{t+1}|S_t=s;\pi} + \gamma \e{\e{G_{t+1}|S_{t+1}=s',\pi}|S_t=s;\pi}  \nn
&=& \e{R_{t+1}|S_t=s;\pi} + \gamma \e{v_\pi(s') |S_t=s;\pi} \nn
&=& \sum_a \Bigg[\e{R_{t+1}|S_t=s,a} + \gamma \e{v_\pi(s') |S_t=s,a} \Bigg] \pi(a|s) \nn
&=& \sum_a \Bigg[\sum_r r p(r|s,a) + \gamma \sum_{s'} v_\pi(s') p(s'|s,a) \Bigg] \pi(a|s) 
\eea

Similarly, we get
\bea
q_\pi(s,a) &=& \e{G_t | S_t=s,A_t=a;\pi} \nn
&=& \e{R_{t+1} |S_t=s,A_t=a;\pi} + \gamma \e{G_{t+1} | S_t=s, A_t=a} \nn
&=& \e{R_{t+1} |S_t=s,A_t=a;\pi} + \gamma \e{ \e{G_{t+1} | S_{t+1}=s',A_{t+1}=a';\pi}|S_t=s, A_t=a;\pi} \nn
&=& \e{R_{t+1} |S_t=s,A_t=a;\pi} + \gamma \e{ q_\pi(s',a')||S_t=s, A_t=a;\pi} \nn
&=& \sum_r r p(r|s,a) + \gamma \sum_{s',a'} q_\pi(s',a') p(s',a'|s,a) \nn
&=&  \sum_r r p(r|s,a) + \gamma \sum_{s',a'} q_\pi(s',a') \pi(a'|s') p(s'|s,a)
\eea

From these we see
\bea
v_\pi(s) &=&  \e{G_t|S_t=s;\pi} \nn
&=& \e{\e{G_t|S_t=s,A_t=a;\pi} } \nn
&=& \e{q_\pi(s,a)} \nn
&=& \sum_a q_\pi(s,a) \pi(a|s)
\eea
Further we have
\bea
q_\pi(s,a) = \sum_r r p(r|s,a) + \gamma \sum_{s'} v_\pi(s') p(s'|s,a)
\eea


We get for the optimal policy
\bea
v_\star(s) &=&\underset{a}{\text{max}}  \Bigg[\sum_r r p(r|s,a) + \gamma \sum_{s'} v_\pi(s') p(s'|s,a) \Bigg]  \\
q_\star(s,a) &=& \sum_r r p(r|s,a)  + \gamma \sum_{s'} \left( \underset{a'}{\text{max}} ~q_\pi(s',a') \right)p(s'|sa)
\eea



\section{Monte Carlo}

\subsection{The idea}

When we do not know the dynamics of the system (i.e. $p(s'|s,a)$ and/or $p(r|s,a)$ then we can sample whole episodes several times to get
\be
v_\pi(s), q_\pi(s,a)
\ee 
by starting off in state $s$ or in state $s$ and take action $a$ and then follow through till the end of the episode. In doing so we can take the first visit approach where we update the averages for a state or state action pair only for the first visit or we can do this for all the visits. Both converse but proving theorems is easier for the former.

\subsection{Importance Sampling}

We have the expectation of a reward under policy $\pi$ 
\bea
\mathbb E_{\pi}[R_{t+1+k}] &=& \sum_{R_{t+1+k}, S_t,A_t, \dots S_{t+k}, A_{t+k}} R_{t+1+k} ~ p(R_{t+1+k}| S_{t+k},A_{t+k}) \nn
&& \times \left( \prod_{n=0}^{k-1} \pi(A_{t+n+1} | S_{t+n+1} ) p( S_{t+n+} | S_{t+n}, A_{t+n}) \right) \nn
&&\times \color{red}{\pi(A_t|S_t)} \color{blue}{p(S_t)} 
\eea
and under a different policy we have
\bea
\mathbb E_{b}[R_{t+1+k}] &=& \sum_{R_{t+1+k}, S_t,A_t, \dots S_{t+k}, A_{t+k}} R_{t+1+k} ~ p(R_{t+1+k}| S_{t+k},A_{t+k}) \nn
&& \times \left( \prod_{n=0}^{k-1} b(A_{t+n+1} | S_{t+n+1} ) p( S_{t+n+1} | S_{t+n}, A_{t+n}) \right) \nn
&&\times \color{red}{b(A_t|S_t)} \color{blue}{p(S_t)} 
\eea
So clearly we have
\bea
\mathbb E_{\pi}[R_{t+1+k}] &=& \mathbb E_{b}[ \left( \prod_{n=0}^{k-1} \frac{\pi(A_{t+n+1} | S_{t+n+1} )}{b(A_{t+n+1} | S_{t+n+1} )} \right) {\color{red}\frac{\pi(A_{t} | S_{t} )}{b(A_{t} | S_{t} )}}   R_{t+1+k}]  \nn
&=&  \mathbb E_{b}[ {\color{olive} \left( \prod_{n=k}^{T-1} \frac{\pi(A_{t+n+1} | S_{t+n+1} )}{b(A_{t+n+1} | S_{t+n+1} )} \right)} \left( \prod_{n=0}^{k-1} \frac{\pi(A_{t+n+1} | S_{t+n+1} )}{b(A_{t+n+1} | S_{t+n+1} )} \right) {\color{red}\frac{\pi(A_{t} | S_{t} )}{b(A_{t} | S_{t} )}}   R_{t+1+k}]  \nn
\eea
Where in the second line we have written the importance sampling contribution from the full path but the part in olive does not matter as it is after the reward. 



\subsection{Taking Averages}

\subsubsection{Ordinary Importance Sampling}

For Monte Carlo we can talk about two kinds of averages across episodes. The first is {\em ordinary importance sampling} 
\bea
V(s) = \frac{\sum_{t \in \mathcal T(s)} \rho_{t:T(t)-1} G_t}{|\mathcal T(s)|}
\eea
The way to interpret this is that $\mathcal T(s)$ is the union of all time steps at which the state was $s$.  If we are discussing first visit MC then this would be restricted to first visit. T(t) is the first time of termination following $t$. $G_t$ are returns pertaining to time step $t$ and $ \rho_{t:T(t)-1}$ are the importance sampling ratios.

To show how this works we consider an example. We show how the values are updated for first visit and every visit MC for the first two episodes. 
\begin{itemize}
\item Episode 1: $\{(s,a),(s',a',R_1),(s,a,R_2),(s_t,R_3\}$ 
\begin{itemize}
\item First Visit
\bea
q(s,a) &=& (R_1 + \gamma R_2+\gamma^2 R_3) \frac{\pi(a'|s') \pi(a|s)}{b(a'|s') b(a|s)} \\
q(s',a') &=& (R_2 + \gamma R_3) \frac{\pi(a|s)}{b(a|s)}
\eea
\item Every Visit
\bea
q(s,a) &=& \frac{(R_1 + \gamma R_2+\gamma^2 R_3) \frac{\pi(a'|s') \pi(a|s)}{b(a'|s') b(a|s)}+R_3}{2} \\
q(s',a') &=& (R_2 + \gamma R_3) \frac{\pi(a|s)}{b(a|s)}
\eea
\end{itemize}
\item  Epsiode 2: $\{(s',a'),(s',a',R_4),(s,\bar a,R_5),(s_t,R_6)\}$
\begin{itemize}
\item First Visit
\bea
q(s,a) &=& (R_1 + \gamma R_2+\gamma^2 R_3) \frac{\pi(a'|s') \pi(a|s)}{b(a'|s') b(a|s)} \\
q(s',a') &=& \frac{(R_2 + \gamma R_3) \frac{\pi(a|s)}{b(a|s)} + (R_4 + \gamma R_5 + \gamma^2 R_6) \frac{\pi(a'|s') \pi(\bar a|s)}{b(a'|s') b(\bar a|s)}}{2} \\
q(s,\bar a) &=& R_6
\eea
\item Every Visit
\bea
q(s,a) &=& \frac{(R_1 + \gamma R_2+\gamma^2 R_3) \frac{\pi(a'|s') \pi(a|s)}{b(a'|s') b(a|s)}+R_3}{2} \\
q(s',a') &=& \frac{(R_2 + \gamma R_3) \frac{\pi(a|s)}{b(a|s)} + (R_4 + \gamma R_5 + \gamma^2 R_6) \frac{\pi(a'|s') \pi(\bar a|s)}{b(a'|s') b(\bar a|s)} + (R_5+ \gamma R_6)  \frac{ \pi(\bar a|s)}{ b(\bar a|s)}}{3} \\
q(s,\bar a) &=& R_6
\eea
\end{itemize}
\end{itemize}

\subsubsection{Ordinary Importance Sampling: Per-importance importance sampling}

At this point it makes sense to {\em re-write the above after dropping the importance sampling ratios after the reward} as they do not make an effect on expectation but increase variance.
 
\begin{itemize}
\item Episode 1: $\{(s,a),(s',a',R_1),(s,a,R_2),(s_t,R_3\}$ 
\begin{itemize}
\item First Visit
\bea
q(s,a) &=& R_1 + \gamma R_2 \frac{\pi(a'|s')}{b(a'|s')} +\gamma^2 R_3 \frac{\pi(a'|s') \pi(a|s)}{b(a'|s') b(a|s)} \\
q(s',a') &=& R_2 + \gamma R_3 \frac{\pi(a|s)}{b(a|s)}
\eea
\item Every Visit
\bea
q(s,a) &=& \frac{R_1 + \gamma R_2 \frac{\pi(a'|s')}{b(a'|s')} +\gamma^2 R_3 \frac{\pi(a'|s') \pi(a|s)}{b(a'|s') b(a|s)} +R_3}{2} \\
q(s',a') &=& R_2 + \gamma R_3 \frac{\pi(a|s)}{b(a|s)}
\eea
\end{itemize}
\item  Episode 2: $\{(s',a'),(s',a',R_4),(s,\bar a,R_5),(s_t,R_6)\}$
\begin{itemize}
\item First Visit
\bea
q(s,a) &=&R_1 + \gamma R_2 \frac{\pi(a'|s')}{b(a'|s')} +\gamma^2 R_3 \frac{\pi(a'|s') \pi(a|s)}{b(a'|s') b(a|s)} \\
q(s',a') &=& \frac{R_2 + \gamma R_3 \frac{\pi(a|s)}{b(a|s)} + R_4 + \gamma R_5 \frac{\pi(a'|s') }{b(a'|s') } + \gamma^2 R_6 \frac{\pi(a'|s') \pi(\bar a|s)}{b(a'|s') b(\bar a|s)}}{2} \\
q(s,\bar a) &=& R_6
\eea
\item Every Visit
\bea
q(s,a) &=&  \frac{R_1 + \gamma R_2 \frac{\pi(a'|s')}{b(a'|s')} +\gamma^2 R_3 \frac{\pi(a'|s') \pi(a|s)}{b(a'|s') b(a|s)} +R_3}{2} \\
q(s',a') &=& \frac{R_2 + \gamma R_3 \frac{\pi(a|s)}{b(a|s)} + R_4 + \gamma R_5 \frac{\pi(a'|s') }{b(a'|s') } + \gamma^2 R_6 \frac{\pi(a'|s') \pi(\bar a|s)}{b(a'|s') b(\bar a|s)} + R_5+ \gamma R_6 \frac{ \pi(\bar a|s)}{ b(\bar a|s)}}{3} \\
q(s,\bar a) &=& R_6
\eea
\end{itemize}
\end{itemize}


\subsubsection{Weighted Importance Sampling}

\bea
V(s) = \frac{\sum_{t \in \mathcal T(s)} \rho_{t:T(t)-1} G_t}{\sum_{t \in \mathcal T(s)} \rho_{t:T(t)-1} }
\eea
The way to interpret this is that $\mathcal T(s)$ is the union of all time steps at which the state was $s$.  If we are discussing first visit MC then this would be restricted to first visit. T(t) is the first time of termination following $t$. $G_t$ are returns pertaining to time step $t$ and $ \rho_{t:T(t)-1}$ are the importance sampling ratios.

To show how this works we consider an example. We show how the values are updated for first visit and every visit MC for the first two episodes. 
\begin{itemize}
\item Episode 1: $\{(s,a),(s',a',R_1),(s,a,R_2),(s_t,R_3\}$ 
\begin{itemize}
\item First Visit
\bea
q(s,a) &=& (R_1 + \gamma R_2+\gamma^2 R_3) \\
q(s',a') &=& (R_2 + \gamma R_3)
\eea
\item Every Visit
\bea
q(s,a) &=& \frac{(R_1 + \gamma R_2+\gamma^2 R_3) \frac{\pi(a'|s') \pi(a|s)}{b(a'|s') b(a|s)}+R_3}{\frac{\pi(a'|s') \pi(a|s)}{b(a'|s') b(a|s)}+1} \\
q(s',a') &=& (R_2 + \gamma R_3) 
\eea
\end{itemize}
\item  Epsiode 2: $\{(s',a'),(s',a',R_4),(s,\bar a,R_5),(s_t,R_6)\}$
\begin{itemize}
\item First Visit
\bea
q(s,a) &=& (R_1 + \gamma R_2+\gamma^2 R_3)  \\
q(s',a') &=& \frac{(R_2 + \gamma R_3) \frac{\pi(a|s)}{b(a|s)} + (R_4 + \gamma R_5 + \gamma^2 R_6) \frac{\pi(a'|s') \pi(\bar a|s)}{b(a'|s') b(\bar a|s)}}{ \frac{\pi(a|s)}{b(a|s)} + \frac{\pi(a'|s') \pi(\bar a|s)}{b(a'|s') b(\bar a|s)}} \\
q(s,\bar a) &=& R_6
\eea
\item Every Visit
\bea
q(s,a) &=& \frac{(R_1 + \gamma R_2+\gamma^2 R_3) \frac{\pi(a'|s') \pi(a|s)}{b(a'|s') b(a|s)}+R_3}{\frac{\pi(a'|s') \pi(a|s)}{b(a'|s') b(a|s)}+1} \\
q(s',a') &=& \frac{(R_2 + \gamma R_3) \frac{\pi(a|s)}{b(a|s)} + (R_4 + \gamma R_5 + \gamma^2 R_6) \frac{\pi(a'|s') \pi(\bar a|s)}{b(a'|s') b(\bar a|s)} + (R_5+ \gamma R_6)  \frac{ \pi(\bar a|s)}{ b(\bar a|s)}}{\frac{\pi(a|s)}{b(a|s)} + \frac{\pi(a'|s') \pi(\bar a|s)}{b(a'|s') b(\bar a|s)} +\frac{ \pi(\bar a|s)}{ b(\bar a|s)}} \\
q(s,\bar a) &=& R_6
\eea
\end{itemize}
\end{itemize}

\subsection{Algorithms}

\subsubsection{Every Visit Algorithms}
Here is the algorithm for on-policy MC prediction.

\begin{tcolorbox}[enhanced,title=On-policy prediction]
Initialize $Q(s,a)=0$ and $C(s,a)=0$ $\forall~s,a$ \\
Loop over episodes: \\
 \phantom{abcd} Using policy $\pi$ generate the sequence $S_0,A_0, R_1,S_1,A_1 \dots S_{T-1}, A_{T-1},R_T,S_T$ \\
 \phantom{abcd} $G=0$ \\
  \phantom{abcd} For $t$ in $(T-1,T-2,\dots 0)$: \\
  \phantom{abcd}   \phantom{abcd}  $G = \gamma G + R_{t+1}$ \\
  \phantom{abcd}   \phantom{abcd}  $Q(S_t,A_t) = \frac{C_(S_t,A_t) Q_(S_t,A_t) +G}{C(S_t,A_t)+1}$ \\
    \phantom{abcd}   \phantom{abcd}  $C(S_t,A_t) += 1$
\end{tcolorbox}

The algorithm for off-policy oridinary importance sampling MC prediction is

\begin{tcolorbox}[enhanced,title=Off-policy ordinary importance sampling prediction]
Initialize $Q(s,a)=0$ and $C(s,a)=0$ $\forall~s,a$ \\
Loop over episodes: \\
 \phantom{abcd} Using policy $b$ generate the sequence  $S_0,A_0, R_1,S_1,A_1 \dots S_{T-1}, A_{T-1},R_T,S_T$ \\
 \phantom{abcd} $G=0$ \\
  \phantom{abcd} $W=1$ \\
  \phantom{abcd} For $t$ in $(T-1,T-2,\dots 0)$ while $W \ne 0$: \\
  \phantom{abcd}   \phantom{abcd}  $G = \gamma G + R_{t+1}$ \\
  \phantom{abcd}   \phantom{abcd}  $Q(S_t,A_t) = \frac{C_(S_t,A_t) Q_(S_t,A_t) +W G}{C(S_t,A_t)+1}$ \\
    \phantom{abcd}   \phantom{abcd}  $C(S_t,A_t) += 1$ \\
    \phantom{abcd}   \phantom{abcd}  $W*= \frac{\pi(A_t|S_t)}{b(A_t|S_t)}$   
\end{tcolorbox}

The algorithm for off-policy weighted importance sampling MC prediction is

\begin{tcolorbox}[enhanced,title=Off-policy weighted importance sampling prediction]
Initialize $Q(s,a)=0$ and $C(s,a)=0$ $\forall~s,a$ \\
Loop over episodes: \\
 \phantom{abcd} Using policy $b$ generate the sequence  $S_0,A_0, R_1,S_1,A_1 \dots S_{T-1}, A_{T-1},R_T,S_T$ \\
 \phantom{abcd} $G,W=0$ \\
  \phantom{abcd} $W=1$ \\
  \phantom{abcd} For $t$ in $(T-1,T-2,\dots 0)$ while $W \ne 0$: \\
  \phantom{abcd}   \phantom{abcd}  $G = \gamma G + R_{t+1}$ \\
  \phantom{abcd}   \phantom{abcd}  $Q(S_t,A_t) = \frac{C_(S_t,A_t) Q_(S_t,A_t) +W G}{C(S_t,A_t)+W}$ \\
    \phantom{abcd}   \phantom{abcd}  $C(S_t,A_t) += W$ \\
    \phantom{abcd}   \phantom{abcd}  $W*= \frac{\pi(A_t|S_t)}{b(A_t|S_t)}$   
\end{tcolorbox}
The algorithm for off-policy per decision oridinary importance sampling MC prediction is

\begin{tcolorbox}[enhanced,title=Off-policy ordinary importance sampling prediction]
Initialize $Q(s,a)=0$ and $C(s,a)=0$ $\forall~s,a$ \\
Loop over episodes: \\
 \phantom{abcd} Using policy $b$ generate the sequence  $S_0,A_0, R_1,S_1,A_1 \dots S_{T-1}, A_{T-1},R_T,S_T$ \\
 \phantom{abcd} $G=0$ \\
  \phantom{abcd} For $t$ in $(T-1,T-2,\dots 0)$: \\
  \phantom{abcd}   \phantom{abcd}  $G = \gamma G + R_{t+1}$ \\
  \phantom{abcd}   \phantom{abcd}  $Q(S_t,A_t) = \frac{C_(S_t,A_t) Q_(S_t,A_t) + G}{C(S_t,A_t)+1}$ \\
    \phantom{abcd}   \phantom{abcd}  $C(S_t,A_t) += 1$ \\
    \phantom{abcd}   \phantom{abcd}  $G*= \frac{\pi(A_t|S_t)}{b(A_t|S_t)}$   
\end{tcolorbox}

\subsection{Example}


Suppose we have a state $s$ and one terminal state $t$. There are only two actions left $L$ and right $R$. The left takes the agent to the terminal state with probability $q$ and return $1$. It takes the agent to the state $s$ with probability $1-q$. Suppose further the target policy has $\pi(L|s)=p$ and the exploration policy has $b(L|s)=\tilde p$. We evaluate the value and variance of the state under the first visit method. Then we have

\bea
\mathbb E_\pi[G_0] &=& pq \sum_{k=0}^\infty (p (1-q))^k \nn
&=& \frac{pq}{1-p(1-q)}
\eea
We also have
\bea
\mathbb E_\pi[G_0^2] &=& pq \sum_{k=0}^\infty (p (1-q))^k \nn
&=& \frac{pq}{1-p(1-q)}
\eea

If we compute this with importance sampling we get
\bea
\mathbb E_b[G_0 \prod_{t=0}^\infty \frac{\pi(A_t|S_t)}{b(A_t|S_t}] &=& \tilde pq \sum_{k=0}^\infty (\tilde p (1-q))^k \left( \frac{p}{\tilde p} \right)^{k+1} \nn
&=& \frac{pq}{1-p(1-q)}
\eea
which is the same as under the target policy.

We also get
\bea
\mathbb E_b[\left(G_0 \prod_{t=0}^\infty \frac{\pi(A_t|S_t)}{b(A_t|S_t}\right)^2] &=& \tilde pq \sum_{k=0}^\infty (\tilde p (1-q))^k \left( \frac{p}{\tilde p} \right)^{2(k+1)} \nn
&=& \frac{qp^2}{\tilde p} \sum_{k=0}^\infty \left( \frac{(1-q) p^2}{\tilde p}\right)^k
\eea
This variance is divergent if $(1-q)p^2 \ge \tilde p$. In Barto and Sutton, they take $q=0.1, p=1$ and $\tilde p=0.5$.


\appendix

\section{Importance Sampling Example}
Note that
\bea
v_\pi(s) = \mathbb E_b [ \rho_{t:T-1} G_t | S_t=s]
\eea
where $\rho_{t:T-1} = \prod_{k=t}^{T-1} \frac{\pi(A_k|S_k)}{b(A_k|S_k)}$ and
\bea
q_\pi(s,a) = \mathbb E_b [ \rho_{t+1:T-1} G_t | S_t=s, A_t=a]
\eea


Suppose we have a sequence $S_0,A_0,R_1,S_1,A_1,R_2,S_2,A_2,R_3,S_3$ where $S_0=S_2=s$ and $A_0=A_2=a$ and $S_1=s',A_1=a'$ then for ordinary importance sampling starting with $Q(S,A)=0$ we get
\bea
q(s,a) &=& \frac{G_2 + G_0 \rho_{1:2}}{2} \nn
&=&  \frac{ R_3 + (R_1 + \gamma R_2 + \gamma^2 R_3) \frac{\pi(A_1|S_1)\pi(A_2|S_2)}{b(A_1|S_1)b(A_2|S_2)}}{2} \\
q(s',a') &=& G_1 \rho_{2:2} \nn
&=& (R_2 + \gamma R_3) \frac{\pi(A_2|S_2)}{b(A_2|S_2)}
\eea
whereas for weighted importance sampling we have

\bea
q(s,a) &=&  \frac{G_2 + G_0 \rho_{1:2}}{1+\rho_{1:2}} \nn
&=&\frac{ R_3 + (R_1 + \gamma R_2 + \gamma^2 R_3) \frac{\pi(A_1|S_1)\pi(A_2|S_2)}{b(A_1|S_1)b(A_2|S_2)}}{1+\frac{\pi(A_1|S_1)\pi(A_2|S_2)}{b(A_1|S_1)b(A_2|S_2)}} \\
q(s',a') &=& G_1 \nn
&=& (R_2 + \gamma R_3)
\eea

Now note that because of causality we have
\bea
\mathbb E_b [ \rho_{t:T-1}R_{t+k} ] = \mathbb E_b [ \rho_{t:t+k-1}R_{t+k} ]
\eea
so we could also simply take

\bea
q(s,a) &=&  \frac{ R_3 + (R_1 + \gamma R_2 \frac{\pi(A_1|S_1)}{b(A_1|S_1)} + \gamma^2 R_3 \frac{\pi(A_1|S_1)\pi(A_2|S_2)}{b(A_1|S_1)b(A_2|S_2)} ) }{2} \\
q(s',a') &=& (R_2 + \gamma R_3 \frac{\pi(A_2|S_2)}{b(A_2|S_2)})
\eea

\end{document}